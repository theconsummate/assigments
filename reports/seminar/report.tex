%%%%%%%%%%%%%%%%%%%%%%%%%%%%%%%%%%%%%%%%%%%%%%%%%%%%%%%%%%%%%%%%%%%
%%%                                                             %%%
%%%      File: report.tex                                       %%%
%%%    Author: Sabine Schulte im Walde                          %%%
%%%   Purpose: Data report for Research Seminar                 %%%
%%%                                                             %%%
%%%%%%%%%%%%%%%%%%%%%%%%%%%%%%%%%%%%%%%%%%%%%%%%%%%%%%%%%%%%%%%%%%%


\documentclass[12pt,a4paper]{article}


\usepackage{a4wide}
\usepackage{times}

\usepackage{natbib}
\usepackage{graphicx}
\graphicspath{ {./} }


\begin{document}


\title{An analysis of the NYC 311 Calls Dataset}
\author{Dhruv Mishra}
\date{\today}

\maketitle


\vspace{+10mm}
\tableofcontents


\vspace{+10mm}
\section{Introduction}
This report explores a small subset (about 250 MB+, as of 2015) of the database of complaints filed by residents of New York City since 2010 via 311 calls. The data considered in this report can be downloaded from \cite{subset}. The full dataset is available at the NYC open data portal \cite{nycdatafull}.

I first encountered this dataset while doing an online course in Python and I really liked it because the data sheds some light on the way of living of the people of New York City.

\subsection{Columns in this dataset}
The complete dataset has 41 columns but the subset considers only 6 of those columns, as described in \cite{nycdatafull}:
\begin{itemize}
    \item Created Date: Date and Time when the complaint was created.
    \item Closed Date: Date and Time when the complaint was closed by responding agency.
    \item Agency: Acronym of responding City Government Agency.
    \item Complaint Type: This is the first level of a hierarchy identifying the topic of the incident or condition. Complaint Type may have a corresponding Descriptor or may stand alone.
    \item Descriptor: This is associated with the Complaint Type, and provides further detail on the incident or condition. Descriptor values are dependent on the Complaint Type and are not always required in a complaint.
    \item City: City of the incident location provided by geovalidation. These cities are actually the different neighbourhoods in NYC (for example Manhattan, Upper East Side, Brooklyn etc.) and should not be confused by the conventional definition of a city.
\end{itemize}

\subsection{Preview of the data}
The top 5 rows of the database look like:

\begin{center}
 \begin{tabular}{||p{3cm} p{3cm} p{1cm} p{3cm} p{3cm} p{1.5cm}||}
 \hline
 CreatedDate & ClosedDate & Agency & ComplaintType & Descriptor & City \\ [0.5ex]
 \hline\hline
 2015-09-15 02:14:04.000000 & None & NYPD & Illegal Parking & Blocked Hydrant & None \\
 \hline
 2015-09-15 02:12:49.000000 & None & NYPD & Noise - Street/Sidewalk & Loud Talking & NEW YORK \\
 \hline
 2015-09-15 02:11:19.000000 & None & NYPD & Noise - Street/Sidewalk & Loud Talking & NEW YORK \\
 \hline
 2015-09-15 02:09:46.000000 & None & NYPD & Noise - Commercial & Loud Talking & BRONX \\
 \hline
 2015-09-15 02:08:01.000000 & 2015-09-15 02:08:18.000000 & DHS & Homeless Person Assistance & Status Call  & NEW YORK \\ [1ex]
 \hline
\end{tabular}
\end{center}

It is interesting to note that there are a couple of None values in the data which will have to be filtered out.

\subsection{Number of distinct values for each column}
The table below shows the number of distinct values present in each of the different columns.
\begin{center}
 \begin{tabular}{||c c||}
 \hline
 Column & Distinct Values \\ [0.5ex]
 \hline\hline
Agency    & 50 \\
Complaint Type    & 200 \\
Descriptor   & 1220 \\
City   & 547 \\ [1ex]
 \hline
\end{tabular}
\end{center}

\section{Analysis}
\subsection{What are the top 25 most common type of complaints?}  \label{top25}
\begin{verbatim}
SELECT ComplaintType as type, COUNT(*) AS freq
    FROM data
    GROUP BY ComplaintType
    ORDER BY -freq
    LIMIT 25
\end{verbatim}

\includegraphics[scale=0.3]{top25complaints}

Heat/Hot Water comes out as the most common type of complaint.

\subsection{Top 10 cities with the largest number of complaints?}  \label{top10cities}
Let's select the top 10 cities which have recorded the maximum number of complaints. As there are some rows which have None in the City, it is necessary to filter them out. Since the data is case sensitive, the following case-insensitive query is used:

\begin{verbatim}
SELECT City AS name, COUNT(*) AS freq
    FROM data
    WHERE name <> 'None'
    GROUP BY name COLLATE NOCASE
    ORDER BY -freq
    LIMIT 10
\end{verbatim}

\begin{center}
 \begin{tabular}{||c c||}
 \hline
 name & freq \\ [0.5ex]
 \hline\hline
BROOKLYN    & 579363 \\
NEW YORK    & 385655 \\
BRONX   & 342533 \\
STATEN ISLAND   & 92509 \\
Jamaica & 46683 \\
Flushing    & 35504 \\
ASTORIA & 31873 \\
Ridgewood   & 21618 \\
Woodside    & 15932 \\
Corona  & 15740 \\ [1ex]
 % 5 & 88 & 788 & 6344 \\ [1ex]
 \hline
\end{tabular}
\end{center}

It is interesting to see that the number of complaints in the top 3 cities Brooklyn, New York and  Bronx is greater than the other cities by an order of magnitude. Looks like people in these cities are the biggest troublemakers!

\subsection{What is the number of complaints by type and by city for the top 7 cities?}
The first step is to get the complaints by type and city for only the top 7 cities.
\begin{verbatim}
SELECT ComplaintType as complaint_type, City as city_name,
    COUNT(*) AS complaint_count
    FROM data
    WHERE city_name IN ('"BROOKLYN", "NEW YORK", "BRONX",
    "STATEN ISLAND", "Jamaica", "Flushing", "ASTORIA"')
    GROUP BY complaint_type, city_name
\end{verbatim}

Now, merging the output of the above query with the result of section \ref{top25} by doing a left join, we get the distribution of the top complaint types in the top 7 cities.

\includegraphics[scale=0.3]{topcomplaintsfortop7cities}

\subsection{What is the distribution of complaints by hour of day?}
\begin{verbatim}
SELECT strftime ('%H', CreatedDate) AS hour, COUNT(*) AS count
    FROM data
    GROUP BY hour
\end{verbatim}

\includegraphics[scale=0.3]{complaintsbyhour}


An unusual aspect of this data is the excessively large number of reports associated with hour 0 (midnight up to but excluding 1 am). The reason is that there are some complaints that are dated but with no associated time, which was recorded in the data as exactly 00:00:00.000 as can be verified by the following SQL query:

\begin{verbatim}
SELECT COUNT(*)
    FROM data
    WHERE strftime ('%H:%M:%f', CreatedDate) = '00:00:00.000'

--- Result --

COUNT(*)
532285
\end{verbatim}


\subsection{What is the most common hour for noise complaints?}
A "noise complaint" is considered to be any complaint string containing the word noise. All dates without an associated time, i.e., a timestamp of 00:00:00.000 have been filtered out.

\begin{verbatim}
SELECT strftime ('%H', CreatedDate) AS hour, COUNT(*) as count
    FROM data
    WHERE strftime ('%H:%M:%f', CreatedDate) <>'00:00:00.000'
    AND ComplaintType LIKE '%noise%'
    GROUP BY hour
\end{verbatim}

\includegraphics[scale=0.3]{noisecomplaintsbyhour}

As can be expected, noise complaints are recorded during the late night hours when people are most likely trying to sleep.

\subsection{Timeseries for top 5 complaints}
The next figure is a line chart for showing the fraction of complaints (y-axis) associated with each hour of the day (x-axis), with each complaint type shown as a differently coloured line. Only the top 5 complaints are showed and complaints with a zero-timestamp have been excluded.

\includegraphics[scale=0.2]{timeseries}

As can be expected, Illegal Parking and Blocked Driveway complaints peak in the morning around 5 am and Street Conditions peak a little bit later around 7 am. New Yorkers are surely early risers!

Heat/Hot Water complaints remain fairly constant throughout the day. The number of such complaints also suggests that most hot water complaints are recorded with a zero timestamp as these were the most common type of complaints as seen in section \ref{top25}

\section{Conclusion}
The data shows the different types of complaints filed by the residents from various parts of New York City at different times of the day. "Heat/Hot Water" is the most common type of complaint whereas the residents of Brooklyn register the maximum number of complaints.

% Bibliography:
\begin{thebibliography}{9}
\bibitem{nycdatafull}
NYC Open Data
\\\texttt{https://nycopendata.socrata.com/Social-Services/311-Service-Requests\\-from-2010-to-Present/erm2-nwe9}


\bibitem{plotly}
Big data analytics using python and sqlite on NYC's 311 complaints since 2003.
\\\texttt{https://plot.ly/ipython-notebooks/big-data-\\analytics-with-pandas-and-sqlite/}

\bibitem{subset}
NYC Open Data
\\\texttt{https://onedrive.live.com/download?cid=FD520DDC6BE92730\&resid\\=FD520DDC6BE92730\%21616\&authkey=AEeP\_4E1uh-vyDE}


\end{thebibliography}

\end{document}


%% --- END OF FILE
